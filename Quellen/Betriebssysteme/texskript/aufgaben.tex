\documentclass[titlepage,12pt, bibtotoc, liststotoc]{scrreprt}
\usepackage[T1]{fontenc}
%\usepackage[cp850]{inputenc}
\usepackage{german}
\usepackage{amssymb}
\usepackage{scrpage}
\setlength{\unitlength}{5mm}

% Seitenstil

\renewcommand{\sectfont}{\rmfamily\bfseries}
\renewcommand{\headfont}{\normalfont}
\renewpagestyle{plain}%
	{(0pt,.5pt)%
	{}{}{NORDAKADEMIE \hfill Betriebssysteme \hfill Prof. Dr.-Ing. J. Brauer}(\textwidth,.5pt)}
	{(\textwidth,.5pt)%
	{}{}{Aufgaben \headmark\hfill Seite \pagemark}(\textwidth,0pt)}%\pagestyle{plain}
\pagestyle{plain}

\begin{document}
\paragraph{Aufgabe 1:} \ \\
Analysieren Sie die folgende Situation: \\
Zwei Benutzerprozesse $p_1$ und $p_2$, die drucken wollen, sprechen
einen Druckerproze"s an, der Druckauftr"age "uber eine Datei annehmen 
kann, die er nach dem FIFO-Prizip abarbeitet:\\
\begin{picture}(13,6)
	\put(3,0){\line(0,1){5}}
	\put(7,0){\line(0,1){5}}
	\multiput(3,1)(0,1){4}{\line(1,0){4}}
	\put(3.5,3.2){1.Druck}
	\put(3.5,2.2){2.Druck}
	\put(9,3.1){\framebox(3,1){out}}
	\put(9,1.1){\framebox(3,1){in}}
	\put(9,3.5){\vector(-1,0){2}}
	\put(9,1.5){\vector(-1,0){2}}
\end{picture}

\begin{minipage}{6cm}
\hspace*{1cm} \\
$p_1$:
\ttfamily
\begin{tabbing}
\hspace*{1cm}\=\hspace{5mm}\=\hspace{5mm}\= \kill
$\vdots$\>	$\vdots$ \\
101\>		lies in; \\
102\>		schreibe `druck1', in \\
103\>		in := in +1 \\
$\vdots$\>	$\vdots$ \\
\end{tabbing}
\end{minipage}
\begin{minipage}{6cm}
\hspace*{1cm} \\
$p_2$: 
\ttfamily
\begin{tabbing}
\hspace*{1cm}\=\hspace{5mm}\=\hspace{5mm}\= \kill
$\vdots$\>	$\vdots$ \\
101\>		lies in; \\
102\>		schreibe `druck2', in; \\
103\>   	in := in +1 \\
$\vdots$\>	$\vdots$ \\
\end{tabbing}
\end{minipage}

Die Prozesse $p_1$ und $p_2$ wollen also beide einen Druckauftrag 
loswerden, indem sie in in die Datei an die Position \texttt{in} schreiben.

\paragraph{Aufgabe 2:} Leser-Schreiber-Problem \\
Viele Prozesse greifen auf eine gemeinsame Datenstruktur zu:
\begin{itemize}
\item	beliebig viele Prozesse d"urfen gleichzeitig lesen
\item	ein schreibender Proze"s braucht exklusiven Zugriff auf die Datenstruktur
\end{itemize}

L"osen Sie dieses Mutual-Exclusion-Problem mithilfe der Ihnen bekannten 
Synchronisationsprimitive.

\paragraph{Aufgabe 3:}
Problem der Proze"ssynchronisation

\begin{picture}(9,6)
	\put(2.5,3.5){\circle{6}}
	\put(2.5,5.5){\circle{0.5}}
	\put(0.5,4){\circle{0.5}}
	\put(4.5,4){\circle{0.5}}
	\put(1.25,1.75){\circle{0.5}}
	\put(3.75,1.75){\circle{0.5}}
	\put(2.3,1.3){$\pitchfork$}
	\put(0.3,2.8){$\pitchfork$}
	\put(4.3,2.8){$\pitchfork$}
	\put(1.05,5.05){$\pitchfork$}
	\put(3.55,5.05){$\pitchfork$}
\end{picture}
\subparagraph{Gesucht:}
eine L"osung des 5-Philosophen-Problems:
\begin{itemize}
\item	kein Philosoph darf verhungern.
\item	eine Gabel darf nur von einem Philosophen zugleich benutzt werden.
\item	nur die Gabeln unmittelbar rechts und links vom Teller d"urfen benutzt werden.
\item	zum Essen braucht man zwei Gabeln.
\end{itemize}

\paragraph{Aufgabe 4:}
das Bankiersproblem\\
Der Bankier einer Kleinstadt hat einer Reihe von Kunden ein Kreditvolumen
einger"aumt.
\subparagraph{Annahme:}
Nicht alle Kunden werden ihre Kredite gleichzeitig voll aussch"opfen.
\subparagraph{Beispiel:}
Er reserviert \$10.000 f"ur vier Kunden, die insgesamt \$22.000 beanspruchen
k"onnten.
\subparagraph{Anfangszustand:}\ \\
\begin{tabular}{c|r|r}
	Kunde & hat & max \\ \hline
	$A$ & \$0 & \$6.000\\
	$B$ & \$0 & \$5.000\\
	$C$ & \$0 & \$4.000\\
	$D$ & \$0 & \$7.000
\end{tabular}
\subparagraph{Gesucht:}
Verklemmungsfreie Zuteilungsstrategie, wenn Kunden von Zeit zu Zeit
Krediterh"ohungen nachfragen.
\subparagraph{Beispiel:}
sicherer Zustand\\
\begin{tabular}{c|r|r}
	$A$ & \$1.000 & \$6.000\\
	$B$ & \$1.000 & \$5.000\\
	$C$ & \$2.000 & \$4.000\\
	$D$ & \$4.000 & \$7.000\\
	& $\sum$ \$8.000 & $\sum$ \$22.000
\end{tabular}\\
$C$ kann noch Kredit gew"ahrt werden.
\subparagraph{Beispiel:}
unsicherer Zustand\\
\begin{tabular}{c|r|r}
	$A$ & \$1.000 & \$6.000\\
	$B$ & \$2.000 & \$5.000\\
	$C$ & \$2.000 & \$4.000\\
	$D$ & \$4.000 & \$7.000\\
	& $\sum$ \$9.000 & $\sum$ \$22.000
\end{tabular}\\
Wie k"onnen solche Zust"ande vermieden werden?


\end{document}