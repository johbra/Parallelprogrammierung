\documentclass[titlepage,12pt, bibtotoc, liststotoc]{scrreprt}
\usepackage[T1]{fontenc}
%\usepackage[cp850]{inputenc}
\usepackage{german}
\usepackage{amssymb}
\usepackage{scrpage}
\setlength{\unitlength}{5mm}

% Seitenstil

\renewcommand{\sectfont}{\rmfamily\bfseries}
\renewcommand{\headfont}{\normalfont}
\renewpagestyle{plain}%
	{(0pt,.5pt)%
	{}{}{NORDAKADEMIE \hfill Betriebssysteme \hfill Prof. Dr.-Ing. J. Brauer}(\textwidth,.5pt)}
	{(\textwidth,.5pt)%
	{}{}{L"osungen \headmark\hfill Seite \pagemark}(\textwidth,0pt)}%\pagestyle{plain}
\pagestyle{plain}

\begin{document}
\paragraph{Aufgabe 1:}
Analysieren Sie die folgende Situation: \\
Zwei Benutzerprozesse $p_1$ und $p_2$, die drucken wollen, sprechen
einen Druckerproze"s an, der Druckauftr"age "uber eine Datei annehmen 
kann, die er nach dem FIFO-Prizip abarbeitet:\\
\begin{picture}(13,6)
	\put(3,0){\line(0,1){5}}
	\put(7,0){\line(0,1){5}}
	\multiput(3,1)(0,1){4}{\line(1,0){4}}
	\put(3.5,3.2){1.Druck}
	\put(3.5,2.2){2.Druck}
	\put(9,3.1){\framebox(3,1){out}}
	\put(9,1.1){\framebox(3,1){in}}
	\put(9,3.5){\vector(-1,0){2}}
	\put(9,1.5){\vector(-1,0){2}}
\end{picture}

\begin{minipage}{6cm}
\hspace*{1cm} \\
$p_1$:
\ttfamily
\begin{tabbing}
\hspace*{1cm}\=\hspace{5mm}\=\hspace{5mm}\= \kill
$\vdots$\>	$\vdots$ \\
101\>		lies in; \\
102\>		schreibe `druck1', in \\
103\>		in := in +1 \\
$\vdots$\>	$\vdots$ \\
\end{tabbing}
\end{minipage}
\begin{minipage}{6cm}
\hspace*{1cm} \\
$p_2$: 
\ttfamily
\begin{tabbing}
\hspace*{1cm}\=\hspace{5mm}\=\hspace{5mm}\= \kill
$\vdots$\>	$\vdots$ \\
101\>		lies in; \\
102\>		schreibe `druck2', in; \\
103\>   	in := in +1 \\
$\vdots$\>	$\vdots$ \\
\end{tabbing}
\end{minipage}

Die Prozesse $p_1$ und $p_2$ wollen also beide einen Druckauftrag 
loswerden, indem sie in in die Datei an die Position \texttt{in} schreiben.

Der Proze"s k"onnte nach \texttt{101} unterbrochen werden. Beide Prozesse lesen
dann denselben Wert f"ur \texttt{in} und schreiben ihre Druckauftr"age in dieselbe
Stelle. Ein Druckauftrag wird dabei "uberschrieben.

\newpage\paragraph{Aufgabe 2:} Leser-Schreiber-Problem \\
Viele Prozesse greifen auf eine gemeinsame Datenstruktur zu:
\begin{itemize}
\item	beliebig viele Prozesse d"urfen gleichzeitig lesen
\item	ein schreibender Proze"s braucht exklusiven Zugriff auf die Datenstruktur
\end{itemize}

L"osen Sie dieses Mutual-Exclusion-Problem mithilfe der Ihnen bekannten 
Synchronisationsprimitive. 
\paragraph{m"ogliche L"osung:}
\ \\

\begin{minipage}[t]{6cm}
Leser:
\ttfamily
\begin{tabbing}
\hspace*{5mm}\=\kill
wiederhole\\
\>	P(s)\\
\>	lesez"ahler := lesez"ahler+1\\
\>	wenn lesez"ahler=1\\
\>	dann down(db)\\
\>	V(s)\\
\>	liesDatenbank\\
\>	P(s)\\
\>	lesez"ahler := lesez"ahler-1\\
\>	wenn lesez"ahler=0\\
\>	dann up(db)\\
\>	V(s)\\
\>	$\vdots$\\
st"andig
\end{tabbing}
\end{minipage}
\begin{minipage}[t]{6cm}
Schreiber:
\ttfamily
\begin{tabbing}
\hspace*{5mm}\=\kill
wiederhole\\
\>	$\vdots$\\
\>	down(db)\\
\>	schreibeDatenbank\\
\>	up(db)\\
\>	$\vdots$\\
st"andig
\end{tabbing}
\end{minipage}

\paragraph{Aufgabe 3:}
Problem der Proze"ssynchronisation

\begin{picture}(9,6)
	\put(2.5,3.5){\circle{6}}
	\put(2.5,5.5){\circle{0.5}}
	\put(0.5,4){\circle{0.5}}
	\put(4.5,4){\circle{0.5}}
	\put(1.25,1.75){\circle{0.5}}
	\put(3.75,1.75){\circle{0.5}}
	\put(2.3,1.3){$\pitchfork$}
	\put(0.3,2.8){$\pitchfork$}
	\put(4.3,2.8){$\pitchfork$}
	\put(1.05,5.05){$\pitchfork$}
	\put(3.55,5.05){$\pitchfork$}
\end{picture}
\subparagraph{Gesucht:}
eine L"osung des 5-Philosophen-Problems:
\begin{itemize}
\item	kein Philosoph darf verhungern.
\item	eine Gabel darf nur von einem Philosophen zugleich benutzt werden.
\item	nur die Gabeln unmittelbar rechts und links vom Teller d"urfen benutzt werden.
\item	zum Essen braucht man zwei Gabeln.
\end{itemize}
\paragraph{5-Philosophen-Problem}
L"osung ohne Verklemmung und ohne Verhungern:

\begin{minipage}[t]{6cm}
Philosoph:
\ttfamily
\begin{tabbing}
\hspace*{5mm}\=\kill
wiederhole\\
\>	denken;\\
\>	P(s);\\
\>	nimm die linke Gabel;\\
\>	nimm die rechte Gabel;\\
\>	essen;\\
\>	lege die linke Gabel hin;\\
\>	lege die rechte Gabel hin;\\
\>	V(s);\\
st"andig
\end{tabbing}
\end{minipage}
\paragraph{Nachteil:}
nur ein Philosoph kann essen
\paragraph{Abhilfe:}
\begin{itemize}
\item	eine Semaphore pro Philosoph
\item	ein Zustand pro Philosoph
	\begin{itemize}
	\item	essend
	\item	hungrig
	\item	denkend
	\end{itemize}
\end{itemize}

\ttfamily
\begin{tabbing}
\hspace*{5mm}\=\hspace{10mm}\=\kill
Philosoph(p) \{p = Identifikation des Philosophen\}\\
\>	wiederhole\\
\>	\>	denken;\\
\>	\>	nimm\_zwei\_Gabeln(p); \{oder warte\}\\
\>	\>	essen;\\
\>	\>	lege\_Gabeln\_hin(p);\\
\>	st"andig;\\
\\
nimm\_zwei\_Gabeln(p)\\
\>	P(mutex);\\
\>	zustand[p] := hungrig;\\
\>	versuche(p); \{beide Gabeln zu nehmen\}\\
\>	V(mutex);\\
\>	P(s[p]);\\
\\
versuche(p)\\
\>	wenn\>	zustand[p] = hungrig\\
\>	\>	und nicht zustand[links(p)] = essend\\
\>	\>	und nicht zustand[rechts(p)] = essend\\
\>	dann\>	zustand[p] := essend\\
\>	\>	V(s[p])\\
\\
lege\_Gabeln\_hin(p)\\
\>	P(mutex)\\
\>	zustand[p] := denkend;\\
\>	versuche(links(p));\\
\>	versuche(rechts(p));\\
\>	V(mutex)
\end{tabbing}
\rmfamily

\paragraph{Aufgabe 4:}
das Bankiersproblem\\
Der Bankier einer Kleinstadt hat einer Reihe von Kunden ein Kreditvolumen
einger"aumt.
\subparagraph{Annahme:}
Nicht alle Kunden werden ihre Kredite gleichzeitig voll aussch"opfen.
\subparagraph{Beispiel:}
Er reserviert \$10.000 f"ur vier Kunden, die insgesamt \$22.000 beanspruchen
k"onnten.
\subparagraph{Anfangszustand:}\ \\
\begin{tabular}{c|r|r}
	Kunde & hat & max \\ \hline
	$A$ & \$0 & \$6.000\\
	$B$ & \$0 & \$5.000\\
	$C$ & \$0 & \$4.000\\
	$D$ & \$0 & \$7.000
\end{tabular}
\subparagraph{Gesucht:}
Verklemmungsfreie Zuteilungsstrategie, wenn Kunden von Zeit zu Zeit
Krediterh"ohungen nachfragen.
\subparagraph{Beispiel:}
sicherer Zustand\\
\begin{tabular}{c|r|r}
	$A$ & \$1.000 & \$6.000\\
	$B$ & \$1.000 & \$5.000\\
	$C$ & \$2.000 & \$4.000\\
	$D$ & \$4.000 & \$7.000\\
	& $\sum$ \$8.000 & $\sum$ \$22.000
\end{tabular}\\
$C$ kann noch Kredit gew"ahrt werden.
\subparagraph{Beispiel:}
unsicherer Zustand\\
\begin{tabular}{c|r|r}
	$A$ & \$1.000 & \$6.000\\
	$B$ & \$2.000 & \$5.000\\
	$C$ & \$2.000 & \$4.000\\
	$D$ & \$4.000 & \$7.000\\
	& $\sum$ \$9.000 & $\sum$ \$22.000
\end{tabular}\\
Wie k"onnen solche Zust"ande vermieden werden?

\paragraph{L"osung:}
\subparagraph{sicherer Zustand}
\begin{itemize}
\item   keine Verklemmung
\item   alle anstehenden (potentiellen maximalen) Anforderungen k"onnen in
	irgendeiner Reihenfolge befriedigt werden
\end{itemize}
\subparagraph{unsicherer Zustand}
= nicht sicher
\paragraph{verbale Beschreibung:}
Der Bankier pr"uft alle anstehenden Anforderungen, ob die Kreditgew"ahrung in
einen unsicheren Zustand f"uhrt oder nicht.
Um festzustellen, ob ein Zustand sicher ist, ist zu pr"ufen, ob f"ur einen oder
mehrere Kunden deren Maximalforderung erf"ullt werden k"onnte.
Wenn das der Fall ist, wird angenommen, da"s deren Kredite zur"uckgezahlt werden.
Jetzt wird f"ur die "ubrigen Kunden (derjenige, der seinem Maximum am n"achsten
ist, zuerst) nach dem gleichen Verfahren gepr"uft, ob deren Maximum gew"ahrt
werden k"onnte, usw.
Wenn alle Kredite zur"uckgezahlt werden k"onnten, ist der Zustand sicher
 
\end{document}