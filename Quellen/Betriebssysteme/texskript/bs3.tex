% bs-Script
% Kapitel 3
%

Trotz der rasanten Entwicklung der Speichertechnologie gilt nach wie vor, daß
\begin{itemize}
	\item  schnelle Speicher eine geringe Kapazität und hohe Kosten pro 
	gespeichertem Bit aufweisen,


	\item  große Speicher langsam sind aber geringe Kosten pro 
	gespeichertem Bit aufweisen.
\end{itemize}

Das Ziel jeder Speicherverwaltung ist es, den Prozessen einen schnellen 
Speicher scheinbar unbegrenzter Kapazität zur Verfügung zu stellen. 
Gegenstand der Speicherverwaltung des Betriebssystems sind die Speichertypen:
\begin{itemize}
\item	Massenspeicher (Plattenspeicher)
\item	Hauptspeicher (Halbleiter-RAMs) 
\end{itemize}
Abbildung \ref{Hauptspeicheraufteilung} zeigt die Aufteilung des Hauptspeichers.

\begin{figure}[tbp]
\setlength{\unitlength}{5mm}
\begin{picture}(20,10)
        \put(0,1){\framebox(6,5){\begin{tabular}{c}Benutzer-\\programm\end{tabular}}}
        \put(0,6){\framebox(6,2){(leer)}}
        \put(0,8){\makebox(6,2){damals:}}
        \put(7,1){\framebox(6,2){Betriebssystem}}
        \put(7,3){\framebox(6,4){\begin{tabular}{c}Benutzer-\\programm\end{tabular}}}
        \put(7,7){\framebox(6,1){(leer)}}
        \put(7,8){\makebox(6,2){früher:}}
        \put(14,1){\framebox(6,1){Betriebssystem}}
        \put(14,2){\framebox(6,4){\begin{tabular}{c}Benutzer-\\programme\end{tabular}}}
        \put(14,6){\framebox(6,1){(leer)}}
        \put(14,7){\framebox(6,1){Gerätetreiber}}
        \put(14,8){\makebox(6,2){heute:}}
        \put(20,7){\makebox(2,1)[l]{$\}$ROM}}
\end{picture}
	\caption{\label{Hauptspeicheraufteilung}  Aufteilung des Hauptspeichers}
\end{figure}

Um einen idealen Speicher, d.h. einen beliebig großen, schnellen und 
obendrein noch billigen Speicher, mit organisatorischen Mittell 
nachzubilden, baut man aus verschiedenen Speicherarten eine Hierarchie 
auf (s. Abbildung \ref{Speicherhierarchie}). Die unterschiedlichen 
Zugriffszeiten resultieren u.a. auch aus unterschiedlichen 
Zugriffsverfahren. Während Registerspeicher, Cache und Hauptspeicher 
wahlfreien Zugriff erlauben, ist bei Massenspeichern sequentieller 
Zugriff erforderlich.

\begin{figure}
\begin{picture}(23,8)
        \put(0,0){\makebox(5,1){1-10ns}}
        \put(0,1){\framebox(5,1){32 words}}
        \put(0,2){\makebox(5,2){Registerspeicher}}
        \put(6,0){\makebox(5,1){1-10ns}}
        \put(6,1){\framebox(5,2){8-256KB}}
        \put(6,3){\makebox(5,2){Cache}}
        \put(12,0){\makebox(5,1){10-100ns}}
        \put(12,1){\framebox(5,3){16-500MB}}
        \put(12,4){\makebox(5,2){Hauptspeicher}}
        \put(18,0){\makebox(5,1){$\approx$10ms}}
        \put(18,1){\framebox(5,6){$\infty$}}
        \put(18,7){\makebox(5,2){Massenspeicher}}
\end{picture}
	\caption{\label{Speicherhierarchie} Speicherhierarchie}
\end{figure}

\section {Pufferspeicher (Cache-Memory)}

Obwohl das Zusammenspiel zwischen Haupt- und Pufferspeicher vollständig 
durch die Hardware gesteuert wird, soll in diesem Abschnitt kurz die 
prinzipielle Pufferspeicher-Organisation 
mit Hilfe eines Assoziativ-Speichers gezeigt werden, da die Prinzipien 
des Zusammenwirkens zwischen Hauptspeicher und virtuellem Speicher 
ähnlich sind. Abbildung \ref{CAM} zeigt 
die heute in Rechnern vorkommenden Speichertypen mit wahlfreiem Zugriff. 
Ein Assoziativspeicher zeichnet sich dadurch aus, daß nicht nur der 
Inhaltsspeicher variabel ist, sondern auch der Adreßspeicher.

\begin{figure}[tbp]
\begin{center}
	\setlength{\unitlength}{4mm}
\begin{picture}(24,7)
        \put(0,6){\makebox(7,1){RAM}}
        \put(0,2){\line(0,1){2}}
        \put(0,4){\line(2,1){2}}
        \put(2,5){\line(0,-1){4}}
        \put(2,1){\line(-2,1){2}}
        \put(2,2){\framebox(1,2)}
        \put(3,1){\framebox(4,4)}
        \put(1,6){\vector(0,-1){1.5}}
        \put(2,3){\vector(1,0){1}}
        \put(3,2.5){\framebox(4,1){Inhalt}}
        \put(5,6){\vector(0,-1){1}}
        \put(5,1){\vector(0,-1){1}}
        \put(9,6){\makebox(5,1){ROM}}
        \put(9,2){\line(0,1){2}}
        \put(9,4){\line(2,1){2}}
        \put(11,5){\line(0,-1){4}}
        \put(11,1){\line(-2,1){2}}
        \put(11,2){\framebox(1,2)}
        \put(12,1){\line(0,1){4}}
        \put(12,5){\line(2,-1){2}}
        \put(14,4){\line(0,-1){2}}
        \put(14,2){\line(-2,-1){2}}
        \put(10,6){\vector(0,-1){1.5}}
        \put(13,1.5){\vector(0,-1){1.5}}
        \put(16,6){\makebox(8,1){CAM}}
        \put(16,1){\framebox(3,4)}
        \put(19,2){\framebox(1,2)}
        \put(20,1){\framebox(4,4)}
        \put(17,6){\vector(0,-1){1}}
        \put(18,6){\vector(0,-1){1}}
        \put(16,2.5){\framebox(3,1){Adresse}}
        \put(19,3){\vector(1,0){1}}
        \put(20,2.5){\framebox(4,1){Inhalt}}
        \put(22,6){\vector(0,-1){1}}
        \put(22,1){\vector(0,-1){1}}
\end{picture}\\
CAM = Contents Addressable Memory (\emph{Assoziativ-Speicher})

\setlength{\unitlength}{4mm}
\begin{picture}(26,16)
        \multiput(0,5)(0,5){2}{\line(1,0){26}}
        \multiput(8,0)(8,0){2}{\line(0,1){14}}
        \put(8,10){\line(-2,1){8}}
        \put(0,10){\makebox(8,4)[tr]{Adreßspeicher }}
        \put(0,10){\makebox(8,4)[bl]{Inhaltsspeicher}}
        \put(8,10){\makebox(6,5){konstant}}
        \put(14,13){\line(1,1){1}}
        \put(15,14){\line(0,-1){3}}
        \put(15,11){\line(-1,1){1}}
        \put(14,12){\line(0,1){1}}
        \put(16,10){\makebox(6,5){variabel}}
        \put(22,11){\framebox(2,3)}
        \put(0,5){\makebox(5,5){konstant}}
        \put(6,6){\line(0,1){3}}
        \put(6,9){\line(1,-1){1}}
        \put(7,8){\line(0,-1){1}}
        \put(7,7){\line(-1,-1){1}}
        \put(0,0){\makebox(5,5){variabel}}
        \put(5,1){\framebox(2,3)}
        \put(9,8){\line(1,1){1}}
        \put(10,9){\line(0,-1){3}}
        \put(10,6){\line(-1,1){1}}
        \put(9,7){\line(0,1){1}}
        \put(10,7){\framebox(1,1)}
        \put(11,6){\line(0,1){3}}
        \put(11,9){\line(1,-1){1}}
        \put(12,8){\line(0,-1){1}}
        \put(12,7){\line(-1,-1){1}}
        \put(12,5){\makebox(4,5){ROM}}
        \put(17,6){\framebox(2,3)}
        \put(19,7){\framebox(1,1)}
        \put(20,6){\line(0,1){3}}
        \put(20,9){\line(1,-1){1}}
        \put(21,8){\line(0,-1){1}}
        \put(21,7){\line(-1,-1){1}}
        \put(22,5){\makebox(4,5){irrelevant}}
        \put(9,3){\line(1,1){1}}
        \put(10,4){\line(0,-1){3}}
        \put(10,1){\line(-1,1){1}}
        \put(9,2){\line(0,1){1}}
        \put(10,2){\framebox(1,1)}
        \put(11,1){\framebox(2,3)}
        \put(13,0){\makebox(3,5){RAM}}
        \put(17,1){\framebox(2,3)}
        \put(19,2){\framebox(1,1)}
        \put(20,1){\framebox(2,3)}
        \put(22,0){\makebox(4,5){CAM}}
\end{picture}

\end{center}
	\caption{\label {CAM} Varianten von Speichern mit wahlfreiem Zugriff}
\end{figure}

Der Hauptspeicher wird durch den Pufferspeicher gepuffert. Das bedeutet, 
daß die aktuell benötigten Daten sich im Pufferspeicher befinden. Wenn 
sichergestellt werden kann, daß die jeweils vom Prozessor benötigten 
Daten rechtzeitig vom Hauptspeicher in den Pufferspeicher gebracht 
werden, wirkt das System Hauptspeicher/Pufferspeicher wie ein Speicher 
mit der Geschwindigkeit des Pufferspeichers und der Kapazität des 
Hauptspeichers. Anders ausgedrückt: Der Pufferspeicher besitzt scheinbar 
(virtuell) die Kapazität des Hauptspeichers. Die Bezeichnung \emph{virtueller 
Speicher} ist hier aber nicht gebräuchlich.

Abbildung \ref {Pufferspeicher} zeigt das Zusammenwirken von Haupt- (RAM) und 
Pufferspeicher (CAM) bei einem Lesezugriff. Zunächst wird versucht, die 
Adresse im Cache zu finden. Bei einem Treffer wird der Inhalt ausgegeben 
(mit 1 bezeichnete Pfeile). Befindet sich die Adresse nicht im Cache, 
wird auf den Hauptspeicher zugegriffen und der Inhalt ausgegeben. 
Außerdem werden Adresse und Inhalt in den Pufferspeicher kopiert 
(mit 2 bezeichnete Pfeile).
 
\begin{figure}[tbp]
    \begin{center}
    	\setlength{\unitlength}{5mm}
\begin{picture}(15,16)
        \put(1,15){\makebox(2,1){Adresse}}
        \put(12,0){\makebox(2,1){Wert}}
        \put(4,10){\line(0,1){2}}
        \put(4,12){\line(2,1){2}}
        \put(6,13){\line(0,-1){4}}
        \put(6,9){\line(-2,1){2}}
        \put(6,10){\framebox(1,2)}
        \put(7,9){\framebox(4,4){RAM}}
        \put(2,2){\framebox(3,3)}
        \put(5,2.5){\framebox(1,2)}
        \put(6,2){\framebox(4,3){CAM}}
        \put(2,15){\line(0,-1){8.5}}
        \put(2,6.5){\vector(1,0){1}}
        \put(3,7){\vector(0,-1){2}}
        \put(2,7.5){\circle*{0.2}}
        \put(2,7.5){\vector(1,0){2}}
        \put(4,8){\vector(0,-1){3}}
        \put(2,14){\circle*{0.2}}
        \put(2,14){\vector(1,0){3}}
        \put(5,15){\vector(0,-1){2.5}}
        \put(9,15){\vector(0,-1){2}}
        \put(9,9){\line(0,-1){2}}
        \put(9,7){\circle*{0.2}}
        \put(9,7){\vector(-1,0){1}}
        \put(9,7){\vector(1,0){3}}
        \put(8,8){\vector(0,-1){3}}
        \put(12,8){\vector(0,-1){8}}
        \put(8,1){\line(0,1){1}}
        \put(8,1){\vector(1,0){4}}
        \put(5.5,2.5){\circle*{0.2}}
        \put(5.5,2.5){\line(0,-1){1.5}}
        \put(5.5,1){\line(-1,0){4.5}}
        \put(1,1){\vector(0,1){2}}
        \put(1.5,6.5){\circle{0.6}}
        \put(1,6){\makebox(1,1){1}}
        \put(10.5,0.5){\circle{0.6}}
        \put(10,0){\makebox(1,1){1}}
        \put(1.5,1.5){\circle{0.6}}
        \put(1,1){\makebox(1,1){2}}
        \put(3.5,14.5){\circle{0.6}}
        \put(3,14){\makebox(1,1){2}}
        \put(10.5,7.5){\circle{0.6}}
        \put(10,7){\makebox(1,1){2}}
\end{picture}
    \end{center}
	\caption{\label {Pufferspeicher} Zusammenwirken zwischen Haupt- und 
	Pufferspeicher}
\end{figure}

Bei Kopieren einer Speicherzelle vom Haupt- in den Pufferspeicher muß 
dort eine Zelle üeberschrieben werden. Hierfür sind verschiedene 
Ersetzungsstrategien gebräuchlich:
\begin{itemize}
\item   FIFO (First In First Out)
\item   LRU  (Least Recently Used)
\end{itemize}

Bei Schreibzugriffen sind zwei Strategien gebräuchlich um die Inhalte von Haupt- und 
Pufferspeicher konsistent zu halten:
\begin{itemize}
\item \emph{copy back}:	Hauptspeicher wird erst beschrieben, wenn die Stelle im
			Pufferspeicher ersetzt werden soll
\item \emph{write through}:	Aktualisierung bei jedem Schreibzugriff
\end{itemize}

Die Effizienz der Pufferspeicher-Organisation hängt von der Trefferquote 
ab, diese wiederum von dem Verhältnis der Kapazitäten von Haupt- und 
Pufferspeicher sowie den Ersetzungsalgorithmen. 

\section{Virtueller Speicher}

Da schon immer die zur Verfügung stehende Hauptspeicherkapazität für 
größere Anwendungen nicht ausreichte, gab es auch schon früher 
Techniken, die Problem zu umgehen. Früher wurde dafür die sogenannte 
\emph{Overlay-Technik}
verwendet, bei der der Programmierer selbst sein Programm in in den 
Speicher passende Stücke (Overlays) zerlegen mußte und für das Ein- und Auslagern 
der Teile selbst sorgen mußte. Ausgelagerte Programmteile befinden sich 
dabei auf einem Hintergrundspeicher (i.d.R. Plattenspeicher).

In modernen Rechenanlagen obliegen diese 
Aufgaben dem Betriebssystem, das hierzu die \emph{virtueller Speichertechnik}
benutzt. Die virtuelle Speichertechnik ist erst mit Mehrprogrammbetrieb effizient 
nutzbar. Erfordert nämlich die Ausführung eines Prozesses das Nachladen 
von Speicherbereichen vom Hintergrundspeicher, was, verglichen mit einem 
Hauptspeicherzugriff, sehr viel Zeit erfordert, so wird die Ausführung 
des Prozesses unterbrochen und die CPU einem anderen rechenwilligen 
Prozeß zugeordnet.

Um die Organisation eines virtuellen Speichers zu verstehen, ist 
zunächst der Begriff des \emph{Adreßraums} zu klären. Im einfachsten 
Fall sind die im Programm verwendeten Adressen genau die, mit denen auf 
den Speicher zugegriffen wird: 	Programm-Adreßraum = Speicher-Adreßraum
	
Bei virtuellen Speichern werden diese Adreßräume voneinander 
entkoppelt, d.h. der Programm-Adreßraum wird in den Speicher-Adreß"-raum 
umgesetzt. Abbildung \ref{Adr-Raum} zeigt schematisch, wie eine 
Programmadresse (virtuelle Adresse) in eine Speicheradresse mithilfe 
einer Adreßumsetzungstabelle umgesetzt wird.

\begin{figure}[tbp]
\begin{center}
	\begin{picture}(20,15)
	\put(1,13){\framebox(3,1)}
	\put(4,13){\framebox(7,1)}
	\put(1,14){\makebox(10,1){\tiny virtuelle Adresse}}
	\put(1,9){\line(0,1){3}}
	\put(4,9){\line(0,1){3}}
	\put(8,9){\line(0,1){3}}
	\put(1,10){\framebox(3,1){5}}
	\put(4,10){\framebox(4,1){1024}}
	\put(8,10){\makebox(7,1){\tiny Adreßumrechnungstabelle}}
	\put(2.5,13.5){\circle*{0,25}}
	\put(2.5,13.5){\vector(0,-1){2.5}}
	\put(10.5,13.5){\circle*{0.25}}
	\put(10.5,13.5){\line(0,-1){5.5}}
	\put(10.5,8){\vector(-1,0){3.5}}
	\put(6,10){\vector(0,-1){1}}
	\put(5,8){\line(-1,0){2.5}}
	\put(2.5,8){\vector(0,-1){3}}
	\put(5,7){\makebox(2,2){$\bigoplus$}}
	\put(1,6){\line(3,1){3}}
	\put(4,7){\line(0,-1){6}}
	\put(4,1){\line(-3,1){3}}
	\put(1,2){\line(0,1){4}}
	\put(1,4){\framebox(3,1){1024}}
	\put(4,4.5){\vector(1,0){1}}
	\put(4,2){\framebox(1,4)}
	\multiput(5,1)(0,1){6}{\framebox(4,1)}
	\put(19,13){\tiny Hintergrundspeicher}
	\multiput(16,0)(2,0){2}{\line(0,1){14}}
	\multiput(16,1)(0,2){6}{\framebox(2,2)}
	\put(16,9.5){\vector(-4,-3){7}}
\end{picture}
\end{center}
	\caption{\label{Adr-Raum} Umsetzung einer virtuellen in eine reale 
	Adresse}
\end{figure}
	
Zu den Aufgaben des Betriebssystems zählt damit auch die Verwaltung des 
Hauptspeichers hinsichtlich belegter und freier Speicherbereiche. Da 
Programme vom Betriebssystem an beliebigen realen Adressen plaziert werden 
können,  müssen Programme mit ihren Daten verschiebbar (\emph{relocatable}) sein.
Das Betriebssystem muß Sicherheit vor Adreßraumverletzung durch Zugriffsschutz 
gewährleisten.

Sicherheit bedeutet hierbei:
\begin{description}
\item[Bereichsschutz]	Programm- und Datenteile dürfen bei der Adressierung
			nicht überschritten werden
\item[Zugriffsschutz]	Schutz vor unzulässigen Zugriffen.
\end{description}
Aufgaben wie Adreßumsetzung, Bereichsschutz und Zugriffsschutz werden vom
Betriebssystem in Zusammenarbeit mit einer Speicherverwaltungseinheit (\emph{Memory
Management Unit}, MMU, Spezialhardware) erledigt.

Bezüglich der Zerlegung von Programmen und Daten in Teile, die dann ein- 
bzw. ausgelagert werden können unterscheidet man grundsätzlich zwei Möglichkeiten:
Zerlegung in \emph{Segmente} oder \emph{Seiten}.

\subsection{Logische Zerlegung in Segmente}
Die Zerlegung in \emph{Segmente} ist an der Programmstruktur orientiert, z.B.
in Prozedursegmente und Datensegmente. Segmente haben variable Längen. 
\paragraph{virtuelle Adressierung:} (eines Bytes)
\begin{itemize}
\item   Angabe einer \emph{Segmentnummer}
\item   Angabe einer \emph{Bytenummer} innerhalb des Segments
\end{itemize}
\paragraph{reale Adressierung:}
\begin{itemize}
\item   Angabe der \emph{Segment-Basisadresse}
\item   hinzuaddieren der Bytenummer
\end{itemize}
Abbildung \ref{Segmente} zeigt den Zusammenhang zwischen logischen 
und physischen Segmenten. 

\begin{figure}[tbp]
\begin{center}
	\begin{picture}(16,14)
	\put(0,13.5){\small logische Strukturierung}
	\put(1,9){\framebox(5,3){Segment 0}}
	\put(0,11){\makebox(1,1){$0$}}
	\put(0,9){\makebox(1,1){$m$}}
	\put(1,3){\framebox(5,5){Segment 1}}
	\put(0,7){\makebox(1,1){$0$}}
	\put(0,3){\makebox(1,1){$n$}}
	\put(1,0){\framebox(5,2){Segment 2}}
	\put(0,1){\makebox(1,1){$0$}}
	\put(0,0){\makebox(1,1){$p$}}
	\put(10,13.5){\small physischer Adreßraum}
	\put(10,10){\framebox(5,2){Segment 2}}
	\put(15.5,11){\tiny $0=$ Segmentbasisadresse}
	\put(10,7){\framebox(5,3)}
	\put(10,4){\framebox(5,3){Segment 0}}
	\put(15.5,6){\tiny $x=$ Segmentbasisadresse}
	\put(10,1){\framebox(5,3)}
	\put(6,1){\line(1,0){1}}
	\put(7,1){\line(0,1){10}}
	\put(7,11){\vector(1,0){3}}
	\put(6,10.5){\line(1,0){2}}
	\put(8,10.5){\line(0,-1){5}}
	\put(8,5.5){\vector(1,0){2}}
\end{picture}
	\caption{\label{Segmente} Abbildung logischer Segmente des virtuellen 
	Adreßraum auf physische Segmente im Hauptspeicher}
\end{center}
\end{figure}

\paragraph{Segmentverwaltung:}
Jedem Segment ist sind die folgenden Informationen zugeordnet:
\begin{itemize}
\item   Segmentbasisadresse
\item   Längenangabe für Bereichsschutz
\item   Zugriffsattribut für Zugriffsschutz
\item   Hinweis, ob das Segment im Hauptspeicher geladen ist oder nicht
\item   Hinweis, ob das Segment im Hauptspeicher verändert wurde (\emph{dirty
        tag})
\end{itemize}
Diese Angaben werden in einem \emph{Segmentdeskriptor} zusammengefaßt. Die 
Segmentdeskriptoren
werden in einer \emph{Segmenttabelle} verwaltet.
\paragraph{Vorteile der Segmentierung:}
\begin{itemize}
\item   Segmente sind \emph{logische} Einheiten, denen spezifische Merkmale
        zugeordnet werden können.
\item   Überlappende Segmente sind durch geeignete Wahl von Basisadressen und
        Längen möglich (z.B. \emph{shared code}, d.h. gemeinsame Nutzung von
        Programm(teil)en durch verschiedene Prozesse).
\item   Die Segmentlänge kann dynamisch verändert werden (z.B.
        \emph{Stack}-Segmente).
\end{itemize}
\paragraph{Nachteile:}
\begin{itemize}
\item   Die variable Länge der Segmente führt zu aufwendiger Verwaltung der
        freien und belegten Bereiche im Hauptspeicher
        (\emph{Freispeicherverwaltung}).
\item   Segmente müssen komplett geladen bzw. ausgelagert werden.
\item   Das Ein- und Auslagern variabel langer Segmente kann zu einer 
ungünstigen Stückelung des freien Speichers im Hauptspeicher führen. 
Das kann zur Folge haben, daß ein großes Segment nicht geladen werden 
kann, weil kein ausreichend großes zusammenhängendes Stück Speicher 
mehr frei ist, obwohl insgesamt noch genügend Speicher vorhanden wäre.
\end{itemize}

\subsection{Physische Zerlegung in Seiten}
Die Verwaltung des Adreßraumes erfolgt hier nach physischen Gesichtspunkten.
Virtueller und realer Adreßraum werden in Bereiche konstanter Größe
(Seiten (\emph{pages}) oder Rahmen (\emph{frames})) zerlegt. Die Größe einer
Seite ist immer eine Zweier-Potenz (zwischen 512 Bytes und 8KB). 
Abbildung \ref{Seiten} zeigt die Zuordnung von Seiten des virtuellen 
Adreßraums zu Rahmen des Hauptspeichers.
 
\begin{figure}[tbp]
\begin{center}
\begin{picture}(25,12)
	\put(0,7){\makebox(5,5){Segmente}}
	\put(5,10){\framebox(5,2){Seite $0$}}
	\put(5,8){\framebox(5,2){Seite $1$}}
	\put(5,6){\framebox(5,2){Seite $2$}}
	\put(5,4){\framebox(5,2){Seite $3$}}
	\put(5,2){\framebox(5,2)}
	\put(5,0){\framebox(5,2){Seite $n$}}
	\put(10,11){\vector(1,0){5}}
	\put(10,9){\line(1,0){2}}
	\put(12,9){\line(0,-1){2}}
	\put(12,7){\vector(1,0){3}}
	\put(10,5){\line(1,0){3}}
	\put(13,5){\line(0,1){4}}
	\put(13,9){\vector(1,0){2}}
	\put(15,10){\framebox(5,2){Seite $0$}}
	\put(15,8){\framebox(5,2){Seite $3$}}
	\put(15,6){\framebox(5,2){Seite $1$}}
	\put(15,4){\framebox(5,2)}
	\put(20,10){\makebox(5,2){Rahmen $0$}}
	\put(20,8){\makebox(5,2){Rahmen $1$}}
	\put(20,6){\makebox(5,2){Rahmen $2$}}
	\put(20,4){\makebox(5,2){Rahmen $3$}}
	\multiput(4.5,12)(0,-5){2}{\line(1,0){0.5}}
	\put(4.5,12){\line(0,-1){5}}
\end{picture}
\end{center}
	\caption{\label{Seiten} Abbildung von virtuellen Seiten auf Rahmen im 
	Hauptspeicher}
\end{figure}

Jede Seite wird durch einen Deskriptor beschrieben:
\begin{itemize}
\item   Seitennummer (Rahmennummer)
\item   Zugriffsattribut
\item   Hinweis, ob die Seite geladen ist
\item   Hinweis, ob die Seite verändert wurde
\end{itemize}

Abbildung \ref {Seitenadressierung} zeigt, wie eine realen 
Hauptspeicheradresse aus Rahmennummer und
Bytenummer zusammensetzt wird.

\begin{figure}[tbp]
\begin{center}
\setlength{\unitlength}{7mm}
\begin{picture}(13,9)
	\put(0,7){\framebox(3,1){Seiten-Nr.}}
	\put(3,7){\framebox(7,1){Byte-Nr.}}
	\put(0,8){\makebox(10,1){virtuelle Adresse}}
	\put(1.5,7){\vector(0,-1){2}}
	\put(8,7){\vector(0,-1){5}}
	\multiput(0,3)(3,0){3}{\line(0,1){3}}
	\put(0,4){\framebox(3,1){Seite $x$}}
	\put(3,4){\framebox(3,1){Rahmen $y$}}
	\put(4.5,4){\vector(0,-1){2}}
	\put(3,1){\framebox(3,1){Rahmen-Nr.}}
	\put(6,1){\framebox(7,1){Byte-Nr.}}
	\put(3,0){\makebox(10,1){reale Adresse}}
\end{picture}
\end{center}
	\caption{\label {Seitenadressierung} Zusammensetzung einer realen 
	Hauptspeicheradresse}
\end{figure}
\paragraph{Vorteile der Aufteilung des Speichers in Seiten:}
\begin{itemize}
\item   Nur die aktuell benötigten Teile des Programms und seiner Daten werden
        im Hauptspeicher gehalten.
\item   Das Problem der Freispeicherverwaltung entfällt.
\end{itemize}
\paragraph{Nachteile:}
\begin{itemize}
\item   Die Zuordnung logischer Merkmale (von Segmenten) muß zu allen
        betroffenen Seiten erfolgen
\item   Eine Seitentabelle ist viel größer als eine Segmenttabelle.
\end{itemize}

Im Zusammenhang mit der virtuellen Speicherverwaltung sind noch folgende 
Begriffe von Bedeutung:
\paragraph{\emph{Working set}:}
Die Menge der einem Prozeß aktuell zugehörigen Seiten (Rahmen) im
Hauptspeicher
\paragraph{\emph{Page fault}:}
Zugriff auf eine Seite, die sich nicht im Hauptspeicher befindet. Ein 
Page fault führt zur Unterbechung des Prozessors und Aktivierung des 
Betriebssystems (Scheduler).
\paragraph{Seitenflattern (\emph{thrashing}):}
Das Betriebssystem lagert aufgrund einer Überlast rechenwilliger Prozesse 
ständig Prozesse aus und wieder ein.

Um die Nachteile zu vermeiden bzw. die Vorteile der Segment- und der 
Seitenaufteilung zu kombinieren, wird häufig eine kombinierte 
Segment-/Seitenverwaltung benutzt, bei der der Speicher zunächst in 
logische, variabel lange Segmente und diese wiederum in Seiten fester 
Länge unterteilt (vgl. z.B. \cite{Liebig}) werden.

\subsection{Algorithmen für die Plazierung variabel langer Segmente im
Hauptspeicher}
In diesem Abschnitt soll kurz auf die Probleme einer 
Freispeicherverwaltung eingegangen werden, wie sie z.B. bei der Verwaltung 
variabel langer Segmente eines virtuellen Speichers oder auf dem Heap im 
Laufzeitsystem einer 
höheren Programmiersprache auftreten. Grundsätzlich kann hierbei die 
Speicheraufteilung in freie und belegte Bereiche nicht vorhergesagt 
werden. Eine mögliche Situation zeigt Abbildung
\ref{Speicheraufteilung}.

\begin{figure}[tbp]
\begin{center}
	\setlength{\unitlength}{4mm}
\begin{picture}(4,15)
	\put(0,14){\makebox(5,1){Speicher:}}
	\put(0,12){\framebox(5,2){Loch}}
	\put(0,9){\framebox(5,3){Segment 1}}
	\put(0,7){\framebox(5,2){Segment 2}}
	\put(0,5){\framebox(5,2){Loch}}
	\put(0,4){\framebox(5,1){Segment 3}}
	\put(0,1){\framebox(5,3){Loch}}
	\multiput(0,0)(5,0){2}{\line(0,1){1}}
\end{picture}
\end{center}
	\caption{\label {Speicheraufteilung} "`Momentaufnahme"' einer 
	Speicheraufteilung}
\end{figure}

Beim Versuch, ein neues Segment im Hauptspeicher zu plazieren, könnten die 
folgenden Situationen auftreten:
\begin{enumerate}
\item   Ein Segment soll plaziert werden, mindestens ein Loch ausreichender
        Größe ist vorhanden.
\item   Kein Loch ist groß genug für das zu plazierende Segment, aber die
        Summe mehrerer Löcher würde ausreichen.
\item   Der freie Platz insgesamt reicht nicht aus, um einen anstehendes
        Segment aufzunehmen.
\end{enumerate}

\paragraph{Fall 1.:}
Es gibt folgende
Strategien:
\begin{enumerate}
\item   Man wähle das kleinste Loch (\emph{best fit})\\
        Vorteil: Große Löcher werden nicht zerlegt.\\
        Nachteil: Fragmentierung in viele kleine Löcher
\item   Man wähle das erste Loch (\emph{first fit})\\
        Vorteil: Diese Methode ist am schnellsten
\item   Man wähle das größte Loch (\emph{worst fit})\\
        Ziel: Löcher sollen möglichst groß sein
\end{enumerate}

\paragraph{Fall 2.:} Es wird Platz geschaffen durch Verschieben der 
Segmente, dabei brauchen nur die Basisadressen der Segmente verändert 
werden.

\paragraph{Fall 3.:} In diesem Fall kann Platz nur durch Auslagern von 
Segmenten zur Verfügung gestellt werden. 
