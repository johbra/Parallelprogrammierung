% bs-Script
% Kapitel 3
%

\paragraph{Aufgabe:}
persistente (dauerhafte) Speicherung von Daten:
\begin{itemize}
\item	Programme
\item	Dokumente
\end{itemize}
Das Dateisystem bestimmt, wie Dateien
\begin{itemize}
\item	benannt werden
\item	strukturiert sind
\item	zugegriffen werden
\item	beschützt werden
\item	implementiert werden
\item	auf dem Datenträger verwaltet werden
\end{itemize}
\section{Dateibenennung}
Eine Datei ist ein Abstraktionsmechanismus für auf Datenträgern gespeicherte
Daten. Wichtigstes Konzept ist die Namensvergabe.

Es gibt spezifische Konventionen hinsichtlich:
\begin{itemize}
\item	Groß/Kleinschreibung
\item	Zeichenvorrat
\item	Länge (8.3)
\item	Struktur (datei.exe)
\end{itemize}
\section{Dateistruktur}
Unterscheidung zwischen logischer und physischer Struktur
\begin{description}
\item{logisch:} z.B. Pascal: \texttt{var f: FILE OF Record}
\item{physisch:}
\begin{itemize}
\item	einfacher Bytestrom. Für das Betriebssystem ist der Bytestrom ohne
	Struktur (UNIX, MS-DOS)\\
	$\to$ Maximale Flexibilität\\
	$\to$ Maximaler Aufwand für Anwendungsprogramme
\item	Satzstruktur
	\begin{itemize}
	\item	Folge von Sätzen (\emph{records}) fester Länge (z.B. 80 Zeichen, 132 Zeichen)\\
		Lese und Schreiboperationen sind satzbezogen
	\item	Folge von Sätzen variabler Länge
	\end{itemize}
\item	Baumstruktur\\
	Datensätze variabler Länge mit Schlüsselfeld als Sortier- und Suchkriterium
	(Großrechner, kommerzielle DV, ISAM-Dateien)		
\end{itemize}
\end{description}
\section{Dateitypen}
\begin{itemize}
\item	reguläre Dateien für Benutzerdaten
\item	Systemdateien (z.B. Dateiverzeichnisse, \emph{directories})
\item	Spezialdateien (E/A-Geräte als \emph{character special files})\\
	z.B.: con, prn, com$x$ etc.
\end{itemize}
Reguläre Dateien
\begin{itemize}
\item	Textdateien (i.d.R. mit Zeilenstruktur) ASCII
\item	Binäre Dateien
\end{itemize}
\section{Zugriffsarten}
\begin{itemize}
\item	sequentieller Zugriff
\item	wahlfreier Zugriff (\emph{random access})
	\begin{itemize}
	\item	durch Angabe der Satzposition innerhalb der Datei
	\item	durch Angabe eines Schlüssels
	\end{itemize}
\end{itemize}
\section{Dateiattribute}
Schutzattribute\\
Flags (\emph{read only, hidden})\\
Satzattribute (Position des Schlüsselfeldes)\\
Erzeugungsdatum, Datum der letzten Änderung, Größe
\section{Dateiverzeichnisse}
Hierarchische Dateisysteme, Verwaltung in Verzeichnissen.
Verzeichnisse können selbst Verzeichnisse enthalten
Arbeisverzeichnis (\emph{working directory})